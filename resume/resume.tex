% !TEX TS-program = xelatex
% !TEX encoding = UTF-8 Unicode
% !Mode:: "TeX:UTF-8"

\documentclass{resume}
\usepackage{zh_CN-Adobefonts_external} % Simplified Chinese Support using external fonts (./fonts/zh_CN-Adobe/)
%\usepackage{zh_CN-Adobefonts_internal} % Simplified Chinese Support using system fonts
\usepackage{linespacing_fix} % disable extra space before next section
\usepackage{cite}

\begin{document}
\pagenumbering{gobble} % suppress displaying page number

\name{赵军旺}

\basicInfo{
  \email{zhjwpku@gmail.com} \textperiodcentered\
  \phone{(+86) 132-6996-5827} \textperiodcentered\
  \github[zhjwpku]{https://github.com/zhjwpku}}

\section{\faGraduationCap\  教育背景}
\datedsubsection{\textbf{北京大学}}{2013年9月 -- 2016年7月}
\textit{硕士}\ 系统结构
\datedsubsection{\textbf{南京理工大学}}{2009年9月 -- 2013年7月}
\textit{学士}\ 计算机科学与技术

\section{\faOpenid\ 工作经历}
\begin{onehalfspacing}
\datedsubsection{\textbf{华为}}{FusionStorage 项目群 \& CloudBU GaussDB for MySQL}
\datedsubsection{\textit{高级工程师}}{2018年4月至今}
\end{onehalfspacing}

\begin{onehalfspacing}
\datedsubsection{\textbf{四达时代}}{研究院}
\datedsubsection{\textit{中级工程师}}{2016年7月 -- 2018年3月}
\end{onehalfspacing}

\section{\faTasks\ 项目经历}
\datedsubsection{\textbf{MySQL 内核开发}}{2020年6月至今}
\textit{C++/InnoDB}
\begin{onehalfspacing}
\begin{itemize}
  \item \href{https://github.com/mysql/mysql-server/pull/314}{Use mmap for binlog}
  \item InnoDB IO 路径性能优化 (Work In Progress)
\end{itemize}
\end{onehalfspacing}

\datedsubsection{\textbf{OBS Index Layer}}{2019年9月 -- 2020年6月}
\textit{C++/MongoDB/RocksDB/RDMA}
\begin{onehalfspacing}
\begin{itemize}
  \item MongoDB 底层对接 DFV
  \item MongoDB 网络层对接 RoCE
\end{itemize}
\end{onehalfspacing}

\datedsubsection{\textbf{参与实现 FusionStorage Persistence Layer 保电内存缓存方案}}{2018年4月 -- 2018年9月}
\textit{C/Cache/Chunk}
\begin{onehalfspacing}
\begin{itemize}
  \item 调研 PMDK libpmemblk 相关实现
  \item 基于原有的缓存方案适配保电内存介质
  \item 保电内存空间的管理,包括 chunk 的申请、释放,掉电恢复等
\end{itemize}
\end{onehalfspacing}

\datedsubsection{\textbf{Operation Console(服务中台)的设计和实现}}{2017年8月 -- 2018年3月}
\textit{Java/SpringBoot/Dubbo}
\begin{onehalfspacing}
\begin{itemize}
  \item 基于原有的 LDAP 设计实现了一套新的 RBAC 系统,作为公司不同业务线的统一入口
  \item 整合各业务线资源,以微服务的方式逐一将其对接到 Operation Console 中
\end{itemize}
\end{onehalfspacing}

\datedsubsection{\textbf{业务解耦及服务快速上云}}{2016年7月 -- 2017年7月}
\textit{Java/SpringBoot/Dubbo/Jenkinsfile/Swagger}
\begin{onehalfspacing}
\begin{itemize}
  \item 制定微服务项目规范(目录结构、接口规范、构建脚本等)并提供一个模板项目供开发人员快速实现一个新的微服务
  \item 搭建一套简易的 CI/CD 平台,依赖 Jenkins 的 Pipeline 模式,每个微服务都快速以 \textit{开发->测试->测试通过->运维->运维通过->上线} 的方式快速发布,极大减少了部署工作量和交流成本
\end{itemize}
\end{onehalfspacing}

\section{\faCodeFork\ 开源贡献}
\begin{onehalfspacing}
  \begin{itemize}
    \item \href{https://github.com/orlabs/orange/pull/138}{Port kong's ring-balancer to orange}
    \item \href{https://github.com/apache/hadoop/pull/1670/files}{HDFS-14925. Rename operation should check nest snapshot}
    \item \href{https://github.com/mysql/mysql-server/pull/313}{Bug\#101450 mysqlbinlog with start-position not show right FDE pos}
    \item \href{https://github.com/mysql/mysql-server/pull/311}{Bug\#101448 when the input of net\_length\_size is 251\, return 3}
  \end{itemize}
  \end{onehalfspacing}

\section{\faWrench\ IT 技能}
% increase linespacing [parsep=0.5ex]
\begin{itemize}[parsep=0.5ex]
  \item 编程语言: C/C++ > Java > Lua > Python
  \item 并发编程: 熟悉事件驱动及网络编程
  \item 常用工具: GDB/Perf/CMake/Gradle/Jenkins/ELK
\end{itemize}

\section{\faCogs\ 其他}
% increase linespacing [parsep=0.5ex]
\begin{itemize}[parsep=0.5ex]
  \item 博客: https://zhjwpku.com
  \item 语言: CET-6 VET/VEWT
  \item GSoC2016(Ceph): \href{https://gist.github.com/zhjwpku/1bc4202671ed06c799f0bf4739416429}{On the wire encryption}
\end{itemize}

\end{document}
