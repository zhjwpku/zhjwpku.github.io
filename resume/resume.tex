% !TEX TS-program = xelatex
% !TEX encoding = UTF-8 Unicode
% !Mode:: "TeX:UTF-8"

\documentclass{resume}
\usepackage{zh_CN-Adobefonts_external} % Simplified Chinese Support using external fonts (./fonts/zh_CN-Adobe/)
%\usepackage{zh_CN-Adobefonts_internal} % Simplified Chinese Support using system fonts
\usepackage{linespacing_fix} % disable extra space before next section
\usepackage{cite}

\begin{document}
\pagenumbering{gobble} % suppress displaying page number

\name{赵军旺}

\basicInfo{
  \email{zhjwpku@gmail.com} \textperiodcentered\
  \phone{(+86) 132-6996-5827} \textperiodcentered\
  \github[zhjwpku]{https://github.com/zhjwpku}}

\section{\faGraduationCap\ 教育背景}
\datedsubsection{\textbf{北京大学}}{2013年9月 -- 2016年7月}
\textit{硕士}\ 系统结构
\datedsubsection{\textbf{南京理工大学}}{2009年9月 -- 2013年7月}
\textit{学士}\ 计算机科学与技术

\section{\faOpenid\ 工作经历}
\begin{onehalfspacing}
  \datedsubsection{\textbf{星环科技}}{基础架构产品部}
  \datedsubsection{\textit{资深数据库内核研发}}{2023年7月至今}
  \end{onehalfspacing}

\begin{onehalfspacing}
\datedsubsection{\textbf{阿里巴巴}}{阿里云数据库产品事业部 OLAP 产品部}
\datedsubsection{\textit{技术专家}}{2021年5月 -- 2023年6月}
\end{onehalfspacing}

\begin{onehalfspacing}
\datedsubsection{\textbf{华为}}{FusionStorage 项目群 \& CloudBU GaussDB for MySQL}
\datedsubsection{\textit{高级工程师}}{2018年4月 -- 2021年5月}
\end{onehalfspacing}

\begin{onehalfspacing}
\datedsubsection{\textbf{四达时代}}{研究院}
\datedsubsection{\textit{研发工程师}}{2016年7月 -- 2018年3月}
\end{onehalfspacing}

\section{\faTasks\ 项目经历}

\datedsubsection{\textbf{Spacture 时空库}}{2023年7月 -- 至今}
\textit{Spacture 时空库产品化及 项目POC。}
\begin{onehalfspacing}
\begin{itemize}
  \item 工作1: HA、备份工具及插件的选型
  \item 工作2: 探索如何通过 Operator 支持 RDS 及基于 PG 的 MPP 分布式数据库的统一部署
  \item 工作3: POC 测案编写及实施
\end{itemize}
\end{onehalfspacing}

\datedsubsection{\textbf{AnalyticDB PostgreSQL 高可用及同城容灾}}{2023年3月 -- 2023年6月}
\textit{同城容灾是专有云项目常见的企业级特性,针对 GPDB 只支持单副本和两副本的架构进行扩展。}
\begin{onehalfspacing}
\begin{itemize}
  \item 主要工作: 修改 FTS 逻辑适配多副本架构,其中一个备升主之后其它备要去和新主建立 replication
  \item 适用场景: 需要高可用的部署场景,常见于同城容灾专有云局点
  \item 其它: 差异化修复(基于 rsync 进行主备数据同步)
\end{itemize}
\end{onehalfspacing}

\datedsubsection{\textbf{AnalyticDB PostgreSQL Iceberg table}}{2022年9月 -- 2023年3月}
\textit{Apache Iceberg 是一个开源的大数据表格式,旨在为数据湖提供一个高效的元数据管理系统。ADB PG 希望借助 Iceberg 的通用表格式来更好地融入大数据生态。}
\begin{onehalfspacing}
\begin{itemize}
  \item 主要工作: 实现 iceberg-c sdk 及 iceberg-fdw
  \item 适用场景: 查询由其它大数据工具生成的数据,将 gpdb 接入大数据生态
  \item 未来规划: 像 snowflake 一样将 iceberg 视为本地表进行查询写入
\end{itemize}
\end{onehalfspacing}

\datedsubsection{\textbf{AnalyticDB PostgreSQL 差异化存储}}{2022年6月 -- 2022年9月}
\textit{AnalyticDB PostgreSQL 云原生 Serverless 版支持秒级付费、按需启停、存算分离、分时弹性等能力,数据存储在 OSS 上,实例运行环境部署 Dadi 缓存系统,增加了存储成本。差异化存储企业级特性充分利用 OSS 不同存储类型的价格为用户进一步节省成本。}
\begin{onehalfspacing}
\begin{itemize}
  \item 工作1: 提供转冷语法,利用 background worker 异步将 oss 对象转为对应的低频访问存储类型
  \item 工作2: 提供建冷表语法,创建表时指定对应的 oss 存储类型,写入链路直接将文件存储为相应的类型
  \item 工作3: 计费,读写低频访问存储的数据量写入 log,管控采集日志完成计费收集
\end{itemize}
\end{onehalfspacing}

\datedsubsection{\textbf{ADB Pipeline Service}}{2021年12月 --2022年5月}
\textit{APS 在 DTS 之外提供了一种低成本数据同步链路,同时提供将 Kafka/SLS 等消息队列中的数据写入 hudi table 的能力,供 ADB 查询分析。}
\begin{onehalfspacing}
\begin{itemize}
  \item 选型 Flink CDC connectors 开发 datastream job,从 RDS 向 ADB 同步数据
  \item 从 SLS 中并行读取数据并写入 hudi table,作为 ADB 湖仓数据入口
  \item 向 Flink CDC 社区贡献动态增删表特性: \href{https://github.com/ververica/flink-cdc-connectors/pull/777}{add new tables to existing cdc job}
\end{itemize}
\end{onehalfspacing}

\datedsubsection{\textbf{AnalyticDB MySQL 存储引擎}}{2021年5月 -- 2021年11月}
\textit{AnalyticDB MySQL 是阿里巴巴自主研发的海量数据实时高并发在线分析云计算服务,可以在毫秒级针对千亿级数据进行即时的多维分析透视和业务探索。}
\begin{onehalfspacing}
\begin{itemize}
  \item 分析 POC 过程中遇到的性能瓶颈并优化
  \item 开发 Z-order 索引特性,加速多维查询分析
\end{itemize}
\end{onehalfspacing}

\datedsubsection{\textbf{TaurusDB 内核研发}}{2020年5月 -- 2021年5月}
\textit{TaurusDB for MySQL 是华为的一款分布式 MySQL 数据库解决方案,SCN (System Change Number) 用于确保数据库中的数据一致性。}
\begin{onehalfspacing}
\begin{itemize}
  \item 使用 mmap 优化 binlog 性能, \href{https://github.com/mysql/mysql-server/pull/314}{Use mmap for binlog}
  \item SCN 项目相关工作: 适配 purge 流程
\end{itemize}
\end{onehalfspacing}

\datedsubsection{\textbf{FusionStorage OBS Index Layer 网络适配 RDMA}}{2019年9月 -- 2020年5月}
\textit{OBS Index Layer 用于存储对象存储的元数据,基于 MongoDB,存储引擎使用 mongo-rocks,并通过实现 plog ENV 来对接持久层。}
\begin{onehalfspacing}
\begin{itemize}
  \item 主要工作: MongoDB transport layer 适配 xnet 库 (RoCE),提高吞吐减少延时 (bypass kernel)
  \item 适用场景: 当返回大量数据到客户端时对性能提升效果明显,例如 OBS 用户查看一个 bucket 中的所有文件名时
\end{itemize}
\end{onehalfspacing}

\datedsubsection{\textbf{FusionStorage Persistence Layer 保电内存缓存方案}}{2018年4月 -- 2019年9月}
\textit{FusionStorage 是华为推出的一款分布式存储系统,支持文件、块、对象等多种存储访问协议。其中 Persistence Layer——存储系统的持久化层——作为存储系统的底层引擎,负责将数据持久化到存储介质中,同时提供数据保护和数据管理等功能,向上提供 plog 接口。}
\begin{onehalfspacing}
\begin{itemize}
  \item 主要工作: 增加了一种新的缓存类型,基于原有的缓存方案适配保电内存介质
  \item 模块功能: Battery Backup Unit 空间的管理,包括 chunk 的申请、释放,掉电恢复等
  \item 适用场景: 存储数据量较小的元数据,能加快写入速度
\end{itemize}
\end{onehalfspacing}

\section{\faCodeFork\ 开源贡献}
\begin{onehalfspacing}
  \begin{itemize}
    \item Patroni
      \begin{itemize}
        \item \href{https://github.com/zalando/patroni/pull/2960}{Abstract MPP handler and decouple it from DCS}
      \end{itemize}
    \item Greenplum
      \begin{itemize}
        \item \href{https://github.com/greenplum-db/gpdb/pulls?q=is%3Apr+author%3A%40me+}{PRs contributed to gpdb}
      \end{itemize}
    \item pg\_cron
      \begin{itemize}
        \item \href{https://github.com/citusdata/pg_cron/pulls?q=is%3Apr+author%3A%40me+}{PRs contributed to pg\_cron}
      \end{itemize}
    \item cron
      \begin{itemize}
        \item \href{https://github.com/vixie/cron/pull/20}{allow '\$' to indicate last day-of-month}
      \end{itemize}
    \item Apache ORC
      \begin{itemize}
        \item \href{https://github.com/apache/orc/pulls?q=is%3Apr+author%3A%40me+}{PRs contributed to Apache ORC}
      \end{itemize}
    \item Flink CDC
      \begin{itemize}
        \item \href{https://github.com/ververica/flink-cdc-connectors/pulls?q=is%3Apr+author%3Azhjwpku+}{PRs contributed to flink-cdc-connectors}
      \end{itemize}
    \item MySQL
      \begin{itemize}
        \item \href{https://github.com/mysql/mysql-server/pull/313}{Bug\#101450 mysqlbinlog with start-position not show right FDE pos}
        \item \href{https://github.com/mysql/mysql-server/pull/311}{Bug\#101448 when the input of net\_length\_size is 251\, return 3}
      \end{itemize}
    \item HDFS
      \begin{itemize}
        \item \href{https://github.com/apache/hadoop/pull/1670/files}{HDFS-14925. Rename operation should check nest snapshot}
      \end{itemize}
    \item Kong
      \begin{itemize}
        \item \href{https://github.com/orlabs/orange/pull/138}{Port kong's ring-balancer to orange}
      \end{itemize}
  \end{itemize}
  \end{onehalfspacing}

\section{\faWrench\ 技能栈}
% increase linespacing [parsep=0.5ex]
\begin{itemize}[parsep=0.5ex]
  \item 编程语言:  C/C++ > Java > Python > Rust > Lua
  \item 并发编程: 熟悉事件驱动模型及网络编程
  \item 常用工具: GDB/Perf/FlameGraph/tmux
\end{itemize}

\section{\faCogs\ 其他}
% increase linespacing [parsep=0.5ex]
\begin{itemize}[parsep=0.5ex]
  \item 博客: https://zhjwpku.com
  \item 语言: CET-6 VET/VEWT
  \item 论文笔记: \href{https://paper-notes.zhjwpku.com/}{paper notes}
  \item GSoC: 两次 Google Summer of Code (X.Org, Ceph) 项目经历
\end{itemize}

\end{document}
