% !TEX TS-program = xelatex
% !TEX encoding = UTF-8 Unicode
% !Mode:: "TeX:UTF-8"

\documentclass{resume}
\usepackage{zh_CN-Adobefonts_external} % Simplified Chinese Support using external fonts (./fonts/zh_CN-Adobe/)
%\usepackage{zh_CN-Adobefonts_internal} % Simplified Chinese Support using system fonts
\usepackage{linespacing_fix} % disable extra space before next section
\usepackage{cite}

\begin{document}
\pagenumbering{gobble} % suppress displaying page number

\name{赵军旺}

\basicInfo{
  \email{zhjwpku@gmail.com} \textperiodcentered\
  \phone{(+86) 132-6996-5827} \textperiodcentered\
  \github[zhjwpku]{https://github.com/zhjwpku}}

\section{\faGraduationCap\ 教育背景}
\datedsubsection{\textbf{北京大学}}{2013年9月 -- 2016年7月}
\textit{硕士}\ 系统结构
\datedsubsection{\textbf{南京理工大学}}{2009年9月 -- 2013年7月}
\textit{学士}\ 计算机科学与技术

\section{\faOpenid\ 工作经历}
\begin{onehalfspacing}
  \datedsubsection{\textbf{星环科技}}{基础架构产品部}
  \datedsubsection{\textit{资深数据库内核研发}}{2023年7月至今}
  \end{onehalfspacing}

\begin{onehalfspacing}
\datedsubsection{\textbf{阿里巴巴}}{阿里云数据库 OLAP 产品部}
\datedsubsection{\textit{技术专家}}{2021年5月 -- 2023年6月}
\end{onehalfspacing}

\begin{onehalfspacing}
\datedsubsection{\textbf{华为}}{IT 产品线 FusionStorage 项目群}
\datedsubsection{\textit{高级工程师}}{2018年4月 -- 2021年5月}
\end{onehalfspacing}

\begin{onehalfspacing}
\datedsubsection{\textbf{四达时代}}{研究院}
\datedsubsection{\textit{研发工程师}}{2016年7月 -- 2018年3月}
\end{onehalfspacing}

\section{\faTasks\ 项目经历}

\datedsubsection{\textbf{Spacture \& Hippo}}{2023年7月 -- 至今}
\textit{负责 Spacture 时空库及 Hippo 向量库内核研发。}
\begin{onehalfspacing}
\begin{itemize}
  \item 工作1: HA、备份工具及插件的选型
  \item 工作2: 探索通过 Operator 支持 RDS 及基于 PG 的 MPP 分布式数据库的统一部署(StackGres)
  \item 工作3: Hippo 支持 DiskANN 索引,基于原生 DiskANN 进行优化,性能提升 30\%
  \item 工作4: Hippo 引入 HNSWlib 代替 Faiss::HNSW,在此基础上设计实现了稠密稀疏联合索引
  \item 工作5: 开发 HippoBench,支持多种开源数据集,统一了 POC 调优、性能优化等场景使用的工具,显著提升了团队工作效率
\end{itemize}
\end{onehalfspacing}

\datedsubsection{\textbf{AnalyticDB PostgreSQL 高可用及同城容灾}}{2022年6月 -- 2023年6月}
\textit{同城容灾是专有云项目常见的企业级特性,针对 GPDB 只支持单副本和两副本的架构进行扩展。}
\begin{onehalfspacing}
\begin{itemize}
  \item 主要工作: 修改 FTS 逻辑适配多副本架构,其中一个备升主之后其它备要去和新主建立 replication
  \item 适用场景: 需要高可用的部署场景,常见于同城容灾专有云局点
  \item 其它: 设计基于实例节点独立 agent 的 HA 解决方案
\end{itemize}
\end{onehalfspacing}

\datedsubsection{\textbf{AnalyticDB PostgreSQL Iceberg table}}{2022年1月 -- 2022年6月}
\textit{Apache Iceberg 是一个开源大数据表格式,旨在为数据湖提供一个高效的元数据管理框架。ADBPG 希望借助 Iceberg 的通用表格式来更好地融入大数据生态。}
\begin{onehalfspacing}
\begin{itemize}
  \item 主要工作: 实现 iceberg-c sdk 及 iceberg-fdw
  \item 适用场景: 查询由其它大数据工具生成的数据,旨在将 gpdb 接入大数据生态
  \item 未来规划: 像 snowflake 一样将 iceberg 视为本地表进行查询写入(适配 table access method)
\end{itemize}
\end{onehalfspacing}

\datedsubsection{\textbf{AnalyticDB 存储引擎支持 Z-order 排序}}{2021年5月 -- 2022年1月}
\textit{AnalyticDB 是阿里巴巴自主研发的海量数据实时高并发在线分析云计算服务,可以在毫秒级针对千亿级数据进行即时的多维分析透视和业务探索。}
\begin{onehalfspacing}
\begin{itemize}
  \item 主要工作: 设计开发 Z-order 索引,build 过程中根据 z-value 对数据进行排序并写入 PartitionEngine
  \item 模块功能: 根据统计信息将不同列在多维查询分析时,过滤不必要的数据块来提升查询分析速度
  \item 适用场景: 查询条件不带 leading column 时,性能优于 cluster index
\end{itemize}
\end{onehalfspacing}

\datedsubsection{\textbf{FusionStorage OBS Index Layer}}{2019年9月 -- 2021年5月}
\textit{OBS Index Layer 用于存储对象存储的元数据,基于 MongoDB,存储引擎使用 mongo-rocks 将数据持久化到 FusionStorage。}
\begin{onehalfspacing}
\begin{itemize}
  \item 工作1: MongoDB transport layer 适配 xnet 库 (RoCE),提高吞吐减少延时 (bypass kernel)
  \item 工作2: mongo-rocks 对接 plog 接口将数据持久化到 FusionStorage 数据底座
  \item 工作3: 针对热点分析,通过合理的批处理、索引优化、参数调整等,提高写入、查询和 Compaction 性能
\end{itemize}
\end{onehalfspacing}

\datedsubsection{\textbf{FusionStorage Persistence Layer 保电内存缓存方案}}{2018年9月 -- 2019年9月}
\textit{FusionStorage 分布式存储系统支持文件、块、对象等多种存储访问协议。Persistence Layer——存储系统的持久化层——作为存储系统的存储引擎,负责将数据持久化到存储介质中,向上提供 plog 接口。}
\begin{onehalfspacing}
\begin{itemize}
  \item 主要工作: 增加一种新的缓存类型,设计数据 Layout 并适配缓存接口
  \item 模块功能: Battery Backup Unit 空间的管理,包括 chunk 的申请、释放,掉电恢复等
  \item 适用场景: 存储数据量较小的元数据,加快写入速度
\end{itemize}
\end{onehalfspacing}

\section{\faCodeFork\ 开源贡献}
\begin{onehalfspacing}
  \begin{itemize}
    \item Patroni
      \begin{itemize}
        \item \href{https://github.com/zalando/patroni/pull/2960}{Abstract MPP handler and decouple it from DCS}
      \end{itemize}
    \item pgvectorBench
      \begin{itemize}
        \item \href{https://github.com/zhjwpku/pgvectorBench}{A lightweight, fast, flexible, and easy-to-use benchmarking tool dedicated to pgvector}
      \end{itemize}
    \item pg\_cron
      \begin{itemize}
        \item \href{https://github.com/citusdata/pg_cron/pull/273}{add possibility to schedule jobs on the last day of month}
        \item \href{https://github.com/citusdata/pg_cron/pulls?q=is%3Apr+author%3A%40me+}{PRs contributed to pg\_cron}
      \end{itemize}
    \item Flink CDC
      \begin{itemize}
        \item \href{https://github.com/ververica/flink-cdc-connectors/pull/777}{add new tables to existing cdc job}
        \item \href{https://github.com/ververica/flink-cdc-connectors/pulls?q=is%3Apr+author%3Azhjwpku+}{PRs contributed to flink-cdc-connectors}
      \end{itemize}
  \end{itemize}
  \end{onehalfspacing}

\section{\faWrench\ 技能栈}
% increase linespacing [parsep=0.5ex]
\begin{itemize}[parsep=0.5ex]
  \item 编程语言:  C/C++ > Rust > Python > Java
  \item 并发编程: 熟悉事件驱动模型及网络编程
  \item 常用工具: GDB/Perf/Valgrind/gperftools
\end{itemize}

\section{\faCogs\ 其他}
% increase linespacing [parsep=0.5ex]
\begin{itemize}[parsep=0.5ex]
  \item 博客: https://zhjwpku.com
  \item 语言: CET-6 VET/VEWT
  \item 论文阅读笔记: \href{https://paper-notes.zhjwpku.com/}{paper notes}
  \item GSoC: 两次 Google Summer of Code (X.Org, Ceph) 项目经历
\end{itemize}

\end{document}
